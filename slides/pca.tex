\documentclass{beamer}
\usetheme{metropolis}
\usecolortheme{dolphin}
\usepackage{amsmath, amsfonts}
\usepackage{graphicx}
\usepackage{tikz}
\usepackage{bm}

\title{Principal Component Analysis (PCA)}
\author{}
\date{}

\begin{document}

%-------------------------------------------
\frame{\titlepage}
%-------------------------------------------

\begin{frame}{Motivating Examples}
\begin{itemize}
    \item Images of digits (e.g., MNIST): Each image is 784-dimensional
    \item Sensors on wearables: 10s–100s of channels, high redundancy
    \item Environmental data: temperature, humidity, air quality across locations
\end{itemize}
\vspace{1em}
\pause
\textbf{Goal:} Reduce dimensions while preserving key structure
\end{frame}

%-------------------------------------------

\begin{frame}{Why PCA? Intuition}
\begin{itemize}
    \item Data lives in high dimensions but often varies in a low-dimensional subspace
    \item PCA finds new axes (principal directions) capturing maximum variance
    \item Reduces noise, saves space, helps visualize
\end{itemize}
\vspace{1em}
\pause
\begin{block}{Key Idea}
Project data onto top-\(k\) directions of highest variance
\end{block}
\end{frame}

%-------------------------------------------

\begin{frame}{Visual Intuition: Elliptical Gaussians}
\centering
\includegraphics[width=0.75\linewidth]{pca_ellipse.png} % Optional: add your plot
\vspace{1em}
\begin{itemize}
    \item Stretch along eigenvectors of covariance
    \item Ellipse axes \( \propto \sqrt{\lambda_1}, \sqrt{\lambda_2} \)
    \item Principal directions = eigenvectors of covariance matrix
\end{itemize}
\end{frame}

%-------------------------------------------

\begin{frame}{Minimal Math}
\textbf{Given:} Data matrix \( X \in \mathbb{R}^{n \times d} \)

\begin{enumerate}
    \item Center the data: \( \mathbf{X}_{\text{centered}} = X - \mu \)
    \item Compute covariance: 
    \[
    \Sigma = \frac{1}{n} X^\top X
    \]
    \item Eigendecompose:
    \[
    \Sigma = U \Lambda U^\top
    \]
    \item Project onto top-\(k\) components:
    \[
    Z = X U_k
    \]
\end{enumerate}
\pause
\textbf{Reconstruction:} \( \hat{X} = Z U_k^\top + \mu \)
\end{frame}

%-------------------------------------------

\begin{frame}{In Code (PyTorch)}
\begin{itemize}
    \item Center data: \texttt{X\_centered = X - X.mean(0)}
    \item Covariance: \texttt{cov = X\_centered.T @ X\_centered / N}
    \item Eigenvectors: \texttt{eigvals, eigvecs = torch.linalg.eigh(cov)}
    \item Project: \texttt{X\_proj = (X @ eigvecs[:, -k:])}
\end{itemize}
\end{frame}

%-------------------------------------------

\begin{frame}{Example: MNIST Digits}
\centering
\includegraphics[width=0.85\linewidth]{mnist_pca_recon.png} % Optional: reconstruction
\vspace{1em}

\textbf{Top component captures most variation.}
\end{frame}

%-------------------------------------------

\begin{frame}{Summary}
\begin{itemize}
    \item PCA finds orthogonal directions of max variance
    \item Works via eigendecomposition of the covariance
    \item Useful for compression, denoising, visualization
\end{itemize}
\pause
\vspace{1em}
\textbf{Next:} PCA for downstream ML tasks
\end{frame}

\end{document}
